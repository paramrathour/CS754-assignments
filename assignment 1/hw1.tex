\title{Assignment 1: CS 754, Advanced Image Processing}
\author{}
\date{Due date: 7th Feb before 11:55 pm}

\documentclass[11pt]{article}

\usepackage{amsmath,soul}
\usepackage{amssymb}
\usepackage{hyperref}
\usepackage{ulem}
\usepackage[margin=0.5in]{geometry}
\begin{document}
\maketitle

\textbf{Remember the honor code while submitting this (and every other) assignment. All members of the group should work on and \emph{understand} all parts of the assignment. We will adopt a \textbf{zero-tolerance policy} against any violation.}
\\
\\
\textbf{Submission instructions:} You should ideally type out all the answers in MS office or Openoffice (with the equation editor) or, more preferably, using Latex. In either case, prepare a pdf file. Create a single zip or rar file containing the report, code and sample outputs and name it as follows: A1-IdNumberOfFirstStudent-IdNumberOfSecondStudent-IdNumberOfThirdStudent.zip. (If you are doing the assignment alone, the name of the zip file is A1-IdNumber.zip. If it is a group of two students, the name of the file should be  A1-IdNumberOfFirstStudent-IdNumberOfSecondStudent.zip). Upload the file on moodle BEFORE 11:55 pm on 7th Feb, which is the time that the submission is due. No assignments will be accepted after a cutoff deadline of 10 am on 8th Feb. Note that only one student per group should upload their work on moodle, although all group members will receive grades. Please preserve a copy of all your work until the end of the semester. \emph{If you have difficulties, please do not hesitate to seek help from me.} The time period between the time the submission is due and the cutoff deadline is to accommodate for any unprecedented issues. But no assignments will accepted after the cutoff deadline. 

\begin{enumerate}
\item Consider a $m \times n$ sensing matrix $\boldsymbol{A}$ ($m < n$)  with mutual coherence $\mu$ and order-$s$ restricted isometry constant (RIC) of $\delta_s$. Prove that $\delta_s \leq \mu(s-1)$. \textsf{[15 points]}

\item Consider a $m \times n$ sensing matrix $\boldsymbol{A}$ ($m < n$) with order-$s$ restricted isometry constant (RIC) of $\delta_s$. Let $\mathcal{S}$ be a subset of up to $s$ elements from $\{1,2,...,n\}$. Let $\boldsymbol{A}_{\mathcal{S}}$ be a $m  \times |S|$ sub-matrix of $\boldsymbol{A}$ with columns corresponding to indices in $\mathcal{S}$. Let $\lambda_{max}$ be the maximum of the maximal eigenvalue of any matrix $\boldsymbol{A}^T_{\mathcal{S}} \boldsymbol{A}_{\mathcal{S}}$ (i.e. the maximum is taken across all possible subsets of size up to $s$). Let $\lambda_{min}$ be the minimum of the minimal eigenvalue of any matrix $\boldsymbol{A}^T_{\mathcal{S}} \boldsymbol{A}_{\mathcal{S}}$ (i.e. the minimum is taken across all possible subsets of size up to $s$). Then prove that $\delta_s = \textrm{max}(1-\lambda_{min},\lambda_{max}-1)$. \textsf{[15 points]}

\item A curious student tries to implement the algorithm implied by the previous question (i.e. question 2). But enumerating all $n^{|S|}$ subsets is too expensive, so the student stops at a total of $10^5$ subsets. Let $\hat{\delta}_S$ be the RIC thus estimated. Then which of the following statements is true? Justify your answer (no credit without justification): (1) $\hat{\delta}_S \leq \delta_S$; (2) $\hat{\delta}_S \geq \delta_S$; (3) $\hat{\delta}_S = \delta_S$; (4) $\hat{\delta}_S < \delta_S$; (5) $\hat{\delta}_S > \delta_S$. \textsf{[15 points]}

\item Let $\boldsymbol{\theta^{\star}}$ be the result of the following minimization problem (BP): $\textrm{min} \|\boldsymbol{\theta}\|_1$ such that $\|\boldsymbol{y}-\boldsymbol{\Phi \Psi \theta}\|_2 \leq \varepsilon$, where $\boldsymbol{y}$ is an $m$-element measurement vector, $\boldsymbol{\Phi}$ is a $m \times n$ measurement matrix ($m < n$), $\boldsymbol{\Psi}$ is a $n \times n$ orthonormal basis in which $n$-element signal $\boldsymbol{x}$ has a sparse representation of the form $\boldsymbol{x} = \boldsymbol{\Psi \theta}$. Notice that $\boldsymbol{y} = \boldsymbol{\Phi x} + \boldsymbol{\eta}$ and $\varepsilon$ is an upper bound on the magnitude of the noise vector $\boldsymbol{\eta}$.

Theorem 5 we studied in class states the following: If $|S| < 0.5(1+1/\mu(\boldsymbol{\Phi \Psi}))$, then we have $\|\boldsymbol{\theta} - \boldsymbol{\theta^{\star}}\|_2 \leq C_1 \|\boldsymbol{\theta}-\boldsymbol{\theta_S}\|_1 + C_2 \varepsilon$ where $C_1$ and $C_2$ are increasing functions of $\mu(\boldsymbol{\Phi \Psi})$ and where $\forall i \in \mathcal{S}, \boldsymbol{\theta_S}_i = \theta_i; \forall i \notin \mathcal{S}, \boldsymbol{\theta_S}_i = 0$. Here $S$ is a set containing the indices of the $s$ largest magnitude elements of $\boldsymbol{\theta}$. 

A curious student asks the following questions: `(1) It appears that the upper bound on $\|\boldsymbol{\theta} - \boldsymbol{\theta^{\star}}\|_2$ is reduced as $|S|$ increases, which goes against the very premise of compressed sensing. How do we address this apparent discrepancy? (2) It also appears that the error bound is independent of $m$. How do you address this? (3) Now consider that I gave you another theorem (called Theorem 5A), which is the same as Theorem 5 except that it requires that $|S| < 0.3333(1+1/\mu(\boldsymbol{\Phi \Psi}))$. Out of Theorem 5 and Theorem 5A, which is the more useful theorem? Why? (4) It appears that if I set $\varepsilon = 0$ in BP, I can always reduce the upper bound on the error even if the noise vector $\boldsymbol{\eta}$ has non-zero magnitude. Am I missing something? If so, what am I missing?'
\\
Your job is to answer all four of the student's questions. \textsf{[5+5+5+5=20 points]}

\item In the first lecture, we looked at several image restoration problems. Recall that image restoration is the process of reversing a degradation effect induced in an image, and the output is the so called `restored image'. We looked at four restoration problems: image denoising (noise removal), image deblurring (removal of blur as well as noise), image inpainting (filling in holes in an image) and reflection removal. Your job is to perform a google search and find out a research paper which tackles any \emph{other} image restoration problem. In your report, write down the title of the paper, the venue and year of publication, and describe the specific restoration problem that is tackled in the paper. Possible venues: conferences such as IEEE CVPR, IEEE ICCV and ECCV; journals such as IEEE TIP, IEEE TCI, IEEE TPAMI, SIAM Journal on Imaging Sciences. \textsf{[10 points]}

\item Construct a synthetic image $\boldsymbol{f}$ of size $32 \times 32$ in the form of a sparse linear combination of $k$ randomly chosen 2D DCT basis vectors. Simulate $m$ compressive measurements of this image in the form $\boldsymbol{y} = \boldsymbol{\Phi} \text{vec}(\boldsymbol{f})$ where $\text{vec}(\boldsymbol{f})$ stands for a vectorized form of $\boldsymbol{f}$, $\boldsymbol{y}$ contains $m$ elements and $\boldsymbol{\Phi}$ has size $m \times 1024$. The elements of $\boldsymbol{\Phi}$ should be independently drawn from a Rademacher matrix (i.e. the values of the entries should independently be $-1$ and $+1$ with probability $0.5$). Your job is to implement the OMP algorithm to recover $\boldsymbol{f}$ from $\boldsymbol{y}, \boldsymbol{\Phi}$ for $k \in \{5,10,20,30,50,100,150,200\}$ and $m \in \{100,200,...,1000\}$. In the OMP iterations, you may assume knowledge of the true value of $k$. Each time, you should record the value of the RMSE given by $\|\text{vec}(\boldsymbol{f}) - \text{vec}(\boldsymbol{\hat{f}})\|_2/\|\text{vec}(\boldsymbol{f})\|_2$. For $k \in \{5,50,200\}$, you should plot a graph of RMSE versus $m$ and plot the reconstructed images with appropriate captions declaring the value of $k,m$. Also plot the ground truth image. For $m \in \{500,700\}$, you should plot a graph of RMSE versus $k$ and plot the reconstructed images with appropriate captions declaring the value of $k,m$. Also plot the ground truth image. Comment on the behaviour of these plots. Repeat all these tasks with the Iterative Hard Thresholding Algorithm, another greedy algorithm from equation (10) of the paper `Iterative Hard Thresholding for Compressed Sensing' which you can find at \url{https://www.sciencedirect.com/science/article/pii/S1063520309000384}. For implementing this algorithm, you should again assume knowledge of the true $k$. A local copy of this paper is also uploaded onto the homework folder. \textsf{[15 + 10 = 25 points]}
\end{enumerate}
\end{document}